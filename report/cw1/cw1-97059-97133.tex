\documentclass[conference]{IEEEtran}
% \IEEEoverridecommandlockouts
% The preceding line is only needed to identify funding in the first footnote. If that is unneeded, please comment it out.
\usepackage{cite}
\usepackage{amsmath,amssymb,amsfonts}
\usepackage{algorithmic}
\usepackage{graphicx}
\usepackage{textcomp}
\usepackage{xcolor}
\def\BibTeX{{\rm B\kern-.05em{\sc i\kern-.025em b}\kern-.08em
    T\kern-.1667em\lower.7ex\hbox{E}\kern-.125emX}}
\begin{document}

\title{Text Analytica\\}

\author{\IEEEauthorblockN{Florian Bauer}
\IEEEauthorblockA{\textit{Department of Computer Science} \\
\textit{University of Bristol}\\
Bristol, United Kingdom \\
ya18048@bristol.ac.uk}
\and
\IEEEauthorblockN{Nathalie Pett}
\IEEEauthorblockA{\textit{Department of Computer Science} \\
\textit{University of Bristol}\\
Bristol, United Kingdom \\
aq18034@bristol.ac.uk}
}

\maketitle

\begin{abstract}
The application can be run online at http://textanalytics.lukaspman.io/.
\end{abstract}

\section{Introduction}
\begin{itemize}
	\item general introduction and overview over report
\end{itemize}

\subsection{Concept}
\begin{itemize}
	\item summarize concept and idea from FA2
\end{itemize}

\subsection{Limitations of the submitted prototype}
\begin{itemize}
	\item explain what we actually implemented and how it is different from the proposed application in FA2 (point to section about improvements)
\end{itemize}

\section{Platform choice}
The considerations detailed below led to choosing to work with Microsoft Azure to remain within the given time scope for the coursework.
\subsection{Setup}
\begin{itemize}
	\item technical issues with Oracle: unintuitive UI (add contributor), service limits
\end{itemize}

\subsection{Documentation and support}
\begin{itemize}
	\item more documentation
	\item more community support
\end{itemize}

\section{System architecture}
\subsection{Infrastructure}
\begin{itemize}
	\item container and Kubernetes (IaaS)
\end{itemize}
 
\subsection{Data storage}
\begin{itemize}
	\item CosmosDB (PaaS); similar to MongoDB: high availability, scaling, ...
\end{itemize}

\subsection{Microservices}
\begin{itemize}
	\item Flask and 3rd party
	\item REST API between front end and back end,  services (back end, analysis, front end)
\end{itemize}

\subsection{Infrastructure as code}
\begin{itemize}
	\item infrastructure can be deployed from a template automatically
\end{itemize}

\section{Scalability}
\subsection{Infrastructure}
\begin{itemize}
	\item Kubernetes and Virtual Node
\end{itemize}

\subsection{Data storage}
\begin{itemize}
	\item PaaS and other region
\end{itemize}

\subsection{Service}
\begin{itemize}
	\item container and microservices
\end{itemize}

\subsection{Load Testing}
\begin{itemize}
	\item include some evidence on how our application scales
\end{itemize}

\section{Future improvements}
\subsection{Infrastructure}
\begin{itemize}
	\item serverless (Lambda, Azure functions),  PaaS
\end{itemize}

\subsection{Service}
\begin{itemize}
	\item add functionality (login, pdf, text recognition, more advanced analysis, save files and retrieve them, ...)
	\item better scoping of microservices
	\item use of more Kubernetes functionalities
\end{itemize}

\subsection{Monitoring}
\begin{itemize}
	\item advanced cloud monitoring
\end{itemize}

\section{Conclusion}
conclusion...

\section*{References}

\end{document}
